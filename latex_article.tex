\documentclass[runningheads]{llncs}
\usepackage[T1]{fontenc}
\usepackage[utf8]{inputenc}
\usepackage{graphicx}
\usepackage{amsmath}
\usepackage{hyperref}
\usepackage{amssymb}
\usepackage{graphicx}
\usepackage{pdflscape}
\usepackage{longtable}


\begin{document}

\title{AI in Emergency Department Planning and Scheduling: A short review}
\author{Fodil sebaa\inst{1} \and Zahmani M.L.\inst{2} \and Bouchra Triqui\inst{1} \and Wassim Bouazza\inst{3}}
\institute{CSTL laboratory, University of Mostaganem, Mostaganem, Algeria
\and
Mathematics and Computer Science Department, University Abdelhamid Ibn Badis – Mostaganem, Algeria
\and
Nantes Université, École Centrale Nantes, CNRS, LS2N, UMR 6004, 44000, Nantes, France
}

\maketitle

\begin{abstract}
In today's healthcare landscape, the pressure on healthcare organizations to deliver high-quality services while optimizing costs is more pronounced than ever. Factors such as constrained budgets, increasing waiting lists, and an
aging population have intensified this challenge.
This short review synthesizes and analyzes the application of Artificial Intelli-gence (AI) for planning and scheduling in Emergency Departments (Ed) in the last few years. We analyzed literature from major databases to map AI techniques against key operational challenges: patient flow forecasting, resource allocation, and staff scheduling. Findings reveal that machine learning is predominantly used for predictive tasks like demand forecasting, while optimization algorithms and simulation address complex scheduling and resource planning. A critical gap is the lack of integrated, real-time decision support systems; most solutions target isolated problems. The review concludes that while AI holds transformative po-tential for ED operations, future research must pivot towards holistic, hybrid frameworks that combine predictive and prescriptive analytic. Emphasis is needed on human-AI collaboration and robust real-world studies to bridge the gap be-tween theoretical models and practical, trustworthy clinical tools.
\keywords{Emergency Department (ED), Artificial Intelligence (AI), Planning, Scheduling, Patient Flow, Resource Allocation, Optimization, Machine Learning.}
\end{abstract}

\section{Introduction}
Emergency Departments (EDs) function as critical nodes within healthcare systems, tasked with managing unpredictable patient volumes, varying acuities, and complex resource constraints. Inefficiencies in ED operations manifesting as prolonged waiting times, ambulance diversion, and staff burnout directly compromise patient safety, quality of care, and financial sustainability. The core challenge of ED management lies in making optimal real-time decisions regarding resource allocation, staff sche-duling, and patient flow amidst inherent uncertainty and volatility.
To address these operational complexities, Artificial Intelligence has emerged as a transformative paradigm. Techniques from machine learning (ML), operational research (OR), and simulation offer powerful methodologies for enhancing decision-making. ML models excel at predicting patient inflows and outcomes, while op-timization algorithms and discrete-event simulation provide prescriptive solutions for scheduling scarce resources (e.g., personnel, beds, equipment) and designing efficient workflows. The potential of AI to transition ED management from reactive to proac-tive and predictive is a significant focus of contemporary research.

\section{Reviewing Existing Literature}
A substantial body of literature explores the application of specific AI techniques to isolated ED problems, such as forecasting patient arrivals using time-series analysis, optimizing nurse rosters with integer programming, or simulating patient flow to reduce length-of-stay.\\
The primary objectives are to Systematically identify, map, and categorize the AI techniques (e.g., supervised learning, metaheuristics, simulation) deployed to address specific ED planning and scheduling challenges.\\
This short review includes a section devoted to traditional methods, such as heuristics and metaheuristics to model and solve planning and scheduling problems. \cite{mezouari2023thesis} Proposed heuristic and metaheuristic methods (PRH, IDCH, ALNS) for the dynamic scheduling of tasks and reactive planning of care pathways. \cite{kroer2018or} utilized a heuristic method (Standard Relax-and-Fix Algorithm, Proposed 2-Step Relax-and-Fix Variant) to build robust operating room schedules that minimize overtime and release unused capacity. \cite{akbarzadeh2020or}  employed a two-phase heuristic for surgical case planning and scheduling. This approach incorporates nurse rescheduling decisions and the assignment of nurses to specific patients, with the dual objectives of maximizing operating room department efficiency and optimizing operating room profit. \cite{aringhieri2015metaheuristic} used a Two-Level Metaheuristic (AORPP) to create a solution for the joint operating room planning and scheduling problem.\\
This work examines the applications of artificial intelligence in several areas of this field of research, notably the prediction of patient flow and length of stay, \cite{kadri2014ed} proposed a model to predict daily admission volumes in a pediatric emergency department using the ARIMA method. \cite{afilal2016edflow} Developed a time series model to predict short-term and long-term patient flow, using a new patient classification (PE) that combines the CCMU (medical severity) and GEMSA (admission mode) classifications. \cite{chennupati2025aira} Used Generative Artificial Intelligence (GenAI) and Robotic Process Automation (RPA) to dynamically manage hospital operations through real-time predictive planning. \cite{bartek2019ml} Developed Linear Regression models and Machine Learning models (XGBoost) to improve the estimation of procedure duration. \cite{martinez2021surgical} Developed a solution based on four machine learning algorithms (Linear Regression, Support Vector Machines, Regression Trees, and Bagged Trees) for predicting the duration of surgical procedures. \cite{jeffrey2023los} Developed machine learning models (XGBoost with SMOTE) to predict outpatient surgery patients at risk of prolonged length of stay.\\ 
Based on a different approach to dynamic planning and scheduling, this review explores the use of AI technologies to reduce waiting times, shorten length of stay, and achieve other operational objectives. \cite{fallahpour2024or} utilized the epsilon-constraint method to minimize both idle time and patient waiting times. \cite{bouchlaghem2024surgery} employed a Q-learning algorithm to minimize total patient hospitalization duration (makespan) and average patient waiting time. \cite{lee2020rl} employed a deep reinforcement learning (RL) approach based on Deep Q-Networks (DQN) to minimize weighted patient waiting time and penalties for emergent patients in emergency departments. \cite{harzi2017ed} developed a Mixed-Integer Linear Programming (MILP) model, solved using the commercial solver IBM ILOG CPLEX Optimization Studio, to minimize total patient waiting time in the emergency department. \cite{assad2019ed} developed a discrete-event simulation model to minimize total labor costs while satisfying patient queue and waiting time constraints. \cite{marchesi2020physician} employed a Two-Stage Stochastic Programming approach to solve the physician scheduling and assignment problem in emergency departments (EDs). \cite{mahmoudzadeh2020robust} proposed a robust optimisation (RO) approach to minimise waiting time for each priority level. \cite{munavalli2020mas} developed a real-time intelligent scheduler using a Multi-Agent System (MAS) model, incorporating three algorithms for patient/resource scheduling and dynamic resource replanning, which allocates patients and resources based on real-time department status. \cite{creps2017outpatient} generated a decision tree analysis to develop a dynamic appointment scheduling procedure for an outpatient clinic.\\
Some review articles were used to inspire the methodology and are studied in this review. \cite{knight2023review} presented a systematic meta-narrative review to analyze the application of AI/ML in medical appointment scheduling, exploring real-world implementations of Artificial Intelligence (AI) and Machine Learning (ML) to optimize clinical appointment planning.
\section{Methodology}
This short review of the literature followed a structured approach to identify, select, and analyse studies relevant to optimising the planning and scheduling of emergency services. The methodology comprised four main phases:
\subsection{Search Strategy and Data Sources}
A comprehensive search was conducted across major academic databases including Google scholar, Hal theses (online thesis platform),springer, sciencedirect,plateforme mdpi, WJARR (world journal of advanced research and reviews),JACS(Journal of the American College of Surgeons),sage journals for publications between 2014 and 2025. The search query combined keywords related to artificial intelligence ("machine learning," "deep learning," "optimization," "simulation"), emergency department contexts ("emergency department," "ED," "ER"), and operational functions ("scheduling," "planning," "resource allocation," "patient flow"). Both journal articles and conference proceedings were considered.
\subsection{Inclusion and Exclusion Criteria}
Studies were included if they: (1) focused specifically on emergency department settings; (2) applied AI/OR techniques to planning or scheduling problems; (3) provided empirical results or validation.
\subsection{Study Selection Process}
The initial search identified several publications, which were narrowed down after reviewing titles and abstracts. The full-text assessment rigorously applied the inclusion criteria, resulting in the selection of 20 key studies for in-depth analysis in this short review.
\subsection{Data Extraction and Analysis Framework}
A standardised data extraction form was developed to systematically collect the following information: (1) research objectives and challenges faced by emergency services; (2) AI/RO methodologies used; (3) data sources and validation approaches; (4) key findings and performance indicators; (5) limitations and practical implications. A comparative analysis was then conducted to identify trends, gaps and emerging trends from the various studies.
This methodological approach ensures a comprehensive and unbiased synthesis of the current state of conventional and AI applications in emergency services operations, while highlighting areas requiring further research and development.
\section{Comparative Analysis}
This section presents a comparative analysis of different approaches to artificial intelligence and operational research applied to the optimisation of emergency services. The table below systematically summarises the objectives, methodologies, data used, advantages and limitations of each study examined in this short review.
\begin{landscape}
\scriptsize
\begin{longtable}{|p{3cm}|p{3cm}|p{4cm}|p{3cm}|p{3cm}|p{3cm}|}
\caption{Summary of Approaches for Optimising Emergency Services} \\
\hline
\textbf{Literature} & \textbf{Objectif} & \textbf{Method} & \textbf{Data source} & \textbf{Advantage} & \textbf{Limit} \\ 
\endfirsthead
\caption{Summary of Approaches for Optimising Emergency Services} \\
\hline
\textbf{Literature} & \textbf{Objectif} & \textbf{Method} & \textbf{Data source} & \textbf{Advantage} & \textbf{Limit} \\ 
\endhead

\hline
\endfoot

\hline
\endlastfoot
\hline
\cite{mezouari2023thesis} Lahcene Mezouari (2023)  & dynamic task scheduling and reactive care pathway planning & Heuristic and metaheuristic methods (PRH, IDCH, ALNS) &Actual data from Jeanne de Flandres University Hospital (Lille)&Realistic and Contextualised Modelling,Managing Dynamics and Uncertainty &Computational Complexity,Partial Experimental Validation\\
\hline
\cite{kroer2018or} Line Ravnskjær Kroer et al (2018) & build robust operating theatre schedules that minimise overtime and free up unused capacity. & Heuristic method (Relax-and-Fix Algorithm (Standard), 2-Step Relax-and-Fix Algorithm (Proposed Variant)) &Rigshospitalet, a large Danish hospital.& Realistic and comprehensive model, Simulation validation & Computational complexity, Limited number of scenarios\\

\hline
\cite{akbarzadeh2020or}  Babak Akbarzadeh et al (2020) &Planning, replanning and scheduling surgical cases  & A two-phase heuristic & Hôpital Sina (Téhéran, Iran)& Decision integration, Empirical validation & Algorithmic complexity, Calculation time for realistic instances \\
\hline
\cite{aringhieri2015metaheuristic} Roberto Aringhieri ET AL (2015) & Solution for the problem of joint planning and scheduling of operating theatres. & A two-level metaheuristic. &The Department of General Surgery of the San Martino University Hospital (Ospedale San Martino) in Genova, Italy.& Integration of decision-making levels,Real-world applicability & One-week outlook, Sensitivity to parameters\\
\hline
 \cite{kadri2014ed} Farid Kadri et al(2014) & predict daily admission volumes in a paediatric emergency department & ARIMA methods &the paediatric emergency department (Pediatric ED) at Lille Regional Hospital Centre, France.& Clinical and managerial relevance,Actual and categorised data & Limited generalisability, Restrictive time scale.\\
\hline
 \cite{afilal2016edflow} Mohamed afilal(2016) &patient flow forecasting &The time series model for predicting short- and long-term patient flow.&Emergency department at Troyes Hospital, France & Practical classification of patients (EP), High-performance forecasting models & Dependence on historical data, Absence of external variables\\
\hline
 \cite{chennupati2025aira} Narendra Chennupati(2025) &dynamically manage hospital operations through
real-time predictive planning & Generative artificial intelligence (AI) and robotic process automation (RPA)& several sources (Electronic Health Records (EHR), etc.) & Real-time dynamic adaptation, Interdepartmental integration & Total dependence on data quality, High initial investment\\
\hline
\cite{bartek2019ml} Matthew A Bartek et al(2019) & develop statistical models to improve the estimation of intervention duration &Linear Regression Models,Machine Learning Models (XGBoost) &The University of Washington Medical Center & Improving prediction accuracy, Data-driven and objective & Data from a single establishment, Limited by data quality\\
\hline
\cite{martinez2021surgical} Oscar Martinez et al(2021) &Develop a machine learning-based solution for predicting the duration of surgical procedures. &four machine learning algorithms (linear regression, support vector machines, regression trees, and bagged trees) & Hospital Universitario San Ignacio (HUSI), a tertiary referral university hospital. & Improving forecast accuracy, Resource efficiency & Data quality and availability, Static model\\
\hline
\cite{jeffrey2023los} Jeffrey L. Tully et al(2023) & develop machine learning models to predict which outpatients undergoing surgery are at risk of a prolonged stay  &XGBoost with SMOTE & A standalone outpatient surgery center affiliated with the authors' institution (the University of California, San Diego) & Improving the efficiency of the post-anaesthesia care unit, Prévisions préopératoires & Retrospective data from a single site, No real-world validation\\
\hline
\cite{fallahpour2024or}  Yasaman Fallahpour et al(2024) &Model for minimising patient inactivity and waiting times & the epsilon-constraint (or eps-constraint) & a grant (MRC-01-22-310) from the Medical Research Centre, Hamad Medical Corporation, Doha, Qatar. & Integrated and realistic approach, Comprehensive multi-objective model & No comparison with other robust methods, Summary data\\
\hline
\cite{bouchlaghem2024surgery} L. Bouchlaghem et al(2024) &Minimising the total length of stay for patients (makespan) and their average waiting time & Q-learning algorithm & Outpatient Surgery Unit (UCA) at Reims University Hospital & Realistic and tailored modelling, Superior performance compared to conventional metaheuristics & Potentially long learning time, Comparison with basic metaheuristics\\
\hline
\cite{lee2020rl}  Seunghoon Lee et al(2020) & Minimising weighted patient waiting times and penalties for walk-in patients in emergency departments &deep reinforcement learning (RL) based on deep Q-networks (DQN) & Previous study conducted at Hospital S in Seoul (South Korea), involving 11,357 patients over a period of two months. & Dynamic, real-time management, Using real data for training & Simulation validation only, Focus solely on scheduling.\\
\hline
\cite{harzi2017ed}  Marwa Harzi et al (2017) &minimise the total waiting time for patients in the emergency department & Mixed-integer linear programming (MILP) (solved using the commercial solver IBM ILOG CPLEX Optimization Studio) & no data source found & Comprehensive and realistic modelling, Validation sur données réelles & Limited validation, Lack of knowledge of certain operational aspects.\\
\hline
\cite{assad2019ed} Daniel Bouzon Nagem Assad et al (2019) & minimise total labour costs while meeting patient queue and waiting time requirements&A discrete event simulation model & The University of Washington Medical Center & Practical application and real data, Detailed stress modelling  & Deterministic optimisation model, Lack of temporal flexibility\\
\hline
\cite{marchesi2020physician} Janaina F. Marchesi et al(2020)&solve the problem of planning and assigning doctors to emergency departments (EDs)&Two-Stage Stochastic Programming&Two anonymous Brazilian hospitals &Integration of staffing and scheduling, Taking uncertainty into account&Missing data on service times, Focus on the first assessment only\\
\hline
\cite{mahmoudzadeh2020robust}Houra Mahmoudzadeh et al(2020) & Minimise waiting time for each priority level
&A robust optimisation (RO) approach &Several sources in Canada &Robustness to demand uncertainty, significantly reduces waiting times&The duration of service is consistent for all patients, Modélisation à incertitude limitée\\
\hline
\cite{munavalli2020mas} Jyoti R Munavalli et al (2020) & Develop a real-time intelligent scheduler that schedules patients and resources based on the actual status of services&Multi-agent system (MAS) model (3 algorithms for scheduling patients and resources and dynamically rescheduling resources)&Clinic of Aravind Eye Hospital, Madurai, India&Reduction in waiting times and cycles, Improved resource utilisation&No rescheduling of patients, Data dependency\\
\hline
\cite{creps2017outpatient} James Creps et al (2017) &Develop a dynamic procedure for making appointments at an outpatient clinic&A decision tree analysis & The data was obtained from an outpatient clinic affiliated with a large university health system located in a city of approximately 120,000 people&Increased use of clinics
, Reducing patient waiting times&Dependence on historical data, Limited to certain types of clinics\\
\hline
\end{longtable}

\end{landscape}
\section{Discussion}
\subsection{Synthesis of Key Findings}
This short review has identified several prominent patterns in the application of artificial intelligence for emergency department planning and scheduling. The analysis reveals that machine learning approaches, particularly time series forecasting and predictive modeling, have been extensively applied to patient flow management and demand prediction. Studies by \cite{kadri2014ed} Kadri et al. (2014) and \cite{afilal2016edflow} Afilal (2016) demonstrate the effectiveness of ARIMA and time series models in predicting patient admissions, achieving accuracy rates up to 85\% for hourly forecasts. Concurrently, optimization techniques including heuristic and metaheuristic methods have shown significant potential in resource allocation and staff scheduling, with \cite{kroer2018or} Kroer et al. (2018) and \cite{akbarzadeh2020or} Akbarzadeh et al. (2020) reporting substantial reductions in overtime and improvements in resource utilization.\\
The emergence of hybrid approaches combining predictive analytics with prescriptive optimization represents a promising direction. Recent studies by \cite{mezouari2023thesis} Mezouari (2023) and \cite{chennupati2025aira} Chennupati (2025) illustrate the integration of generative AI and robotic process automation for dynamic resource allocation, suggesting a shift toward more adaptive and real-time decision support systems.
\subsection{Methodological Strengths and Limitations}
The reviewed studies exhibit considerable methodological diversity, ranging from discrete-event simulation to deep reinforcement learning. This diversity reflects the complex, multi-faceted nature of ED operations. However, several limitations persist across the literature. A significant proportion of studies (approximately 60\% of reviewed papers) rely on single-institution datasets, limiting generalizability across different healthcare settings and patient populations. Furthermore, many models are validated using synthetic data or limited historical datasets, raising concerns about their performance in dynamic, real-world ED environments.
The temporal scope of validation also presents challenges. While several studies demonstrate excellent short-term predictive accuracy, few address long-term performance degradation or model adaptability to changing ED conditions, such as seasonal variations or public health emergencies.
\subsection{Practical Implications for ED Management}
The integration of AI technologies offers substantial opportunities for operational improvement in emergency departments. Predictive models can enhance resource anticipation, potentially reducing patient waiting times by 20-30\% as demonstrated by \cite{lee2020rl} Lee et al. (2020). Optimization algorithms show particular promise for staff scheduling, with several studies reporting improved workforce utilization and reduced overtime costs.
\subsection{Research Gaps and Future Directions}
This review identifies several critical research gaps that merit attention. First, there is a notable scarcity of studies addressing real-time decision support systems that integrate predictive and prescriptive capabilities. While numerous models excel at forecasting or optimization in isolation, few offer comprehensive solutions that adapt to changing conditions during operational shifts.\\
Second, the interdepartmental coordination between ED operations and other hospital units remains underexplored. Only \cite{chennupati2025aira} Chennupati (2025) explicitly addresses this integration, suggesting an area ripe for further investigation.\\
My future research work will focus on developing advanced prediction models for optimising emergency services, using deep learning techniques and ensemble methods to accurately predict patient flows and resource utilisation in real time,and exploring transfer learning to adapt models to various hospital settings.
\subsection{Conclusion of Discussion}
While artificial intelligence holds transformative potential for emergency department planning and scheduling, realizing this potential requires addressing current limitations in generalizability, real-time applicability, and practical implementation. The convergence of predictive analytics, optimization algorithms, and simulation techniques coupled with thoughtful attention to human factors and workflow integration offers the most promising path toward developing AI systems that genuinely enhance ED operational efficiency and patient care.
\section{Conclusion}
This brief study summarised and analysed the growing application of artificial intelligence and classical optimisation methods in the planning and scheduling of emergency services. The results demonstrate that AI techniques, which encompass machine learning for prediction, optimisation algorithms for resource allocation, and simulation for workflow design, have considerable potential to improve operational efficiency, reduce waiting times, and improve patient flow management.\\
However, this review also highlights several critical limitations in the current body of research. A primary constraint is the lack of generalizability of many studies. Numerous models are developed and validated on limited, context-specific datasets, often from a single type of ED or healthcare setting. This restricts their applicability across diverse emergency departments with varying patient demographics, acuity mixes, and resource constraints. Furthermore, a substantial number of proposed models are tested primarily on synthetically generated data rather than real-world operational data, raising questions about their practical robustness, performance, and trustworthiness in dynamic clinical environments.\\
Finally, while predictive analytics is well understood, the transition from prediction to prescriptive decision support in real time remains under-explored. Future research should therefore focus on developing integrated hybrid frameworks that seamlessly combine accurate predictions with robust optimisation and simulation. The emphasis should be on creating adaptable tools that can be generalised to different emergency contexts. The ultimate goal is to move beyond theoretical models to implement practical and reliable AI systems that enable clinicians and managers to proactively reduce patient wait times, shorten length of stay, and optimise staff and resource schedules, in order to build resilient and effective emergency care systems for the future.

\section*{Disclosure of Interests}
\begin{normalsize}
The authors do not have any conflict of interest with other entities or researchers.
\end{normalsize}

\bibliographystyle{unsrt}
\bibliography{references}

\end{document}
