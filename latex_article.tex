\documentclass[runningheads]{llncs}
\usepackage[T1]{fontenc}
\usepackage[utf8]{inputenc}
\usepackage{graphicx}
\usepackage{amsmath}
\usepackage{hyperref}

\begin{document}

\title{AI in Emergency Department Planning and Scheduling: A short review}
\author{Fodil sebaa\inst{1} \and Zahmani mohamed lamine\inst{2}}
\institute{CSTL laboratory, University of Mostaganem, Mostaganem, Algeria \\
\email{prenom.nom@univ.xyz} \and
Deuxième institution, Ville, Pays \\
\email{deuxieme.auteur@exemple.com}}

\maketitle

\begin{abstract}
This short review synthesizes and analyzes the application of Artificial Intelli-gence (AI) for planning and scheduling in Emergency Departments (Ed) in the last few years. We analyzed literature from major databases to map AI techniques against key operational challenges: patient flow forecasting, resource allocation, and staff scheduling. Findings reveal that machine learning is predominantly used for predictive tasks like demand forecasting, while optimization algorithms and simulation address complex scheduling and resource planning. A critical gap is the lack of integrated, real-time decision support systems; most solutions target isolated problems. The review concludes that while AI holds transformative po-tential for ED operations, future research must pivot towards holistic, hybrid frameworks that combine predictive and prescriptive analytic. Emphasis is needed on human-AI collaboration and robust real-world studies to bridge the gap be-tween theoretical models and practical, trustworthy clinical tools.
\keywords{Emergency Department (ED), Artificial Intelligence (AI), Planning, Scheduling, Patient Flow, Resource Allocation, Optimization, Machine Learning.}
\end{abstract}

\section{Introduction}
Emergency Departments (EDs) function as critical nodes within healthcare systems, tasked with managing unpredictable patient volumes, varying acuities, and complex resource constraints. Inefficiencies in ED operations manifesting as prolonged waiting times, ambulance diversion, and staff burnout directly compromise patient safety, quality of care, and financial sustainability. The core challenge of ED management lies in making optimal real-time decisions regarding resource allocation, staff sche-duling, and patient flow amidst inherent uncertainty and volatility.
To address these operational complexities, Artificial Intelligence has emerged as a transformative paradigm. Techniques from machine learning (ML), operational research (OR), and simulation offer powerful methodologies for enhancing decision-making. ML models excel at predicting patient inflows and outcomes, while op-timization algorithms and discrete-event simulation provide prescriptive solutions for scheduling scarce resources (e.g., personnel, beds, equipment) and designing efficient workflows. The potential of AI to transition ED management from reactive to proac-tive and predictive is a significant focus of contemporary research.

\section{Reviewing Existing Literature and Identifying the Gap}
A substantial body of literature explores the application of specific AI techniques to isolated ED problems, such as forecasting patient arrivals using time-series analysis, optimizing nurse rosters with integer programming, or simulating patient flow to reduce length-of-stay.
This study aims to address this gap by conducting a systematic literature review (SLR) of AI applications for planning and scheduling in Emergency Departments. The primary objectives are to Systematically identify, map, and categorize the AI techniques (e.g., supervised learning, metaheuristics, simulation) deployed to address specific ED planning and scheduling challenges.
The papers  addresses several components of this methodology, notably the prediction of patient flow and length of stay, \cite{kadri2014ed} proposed a model to predict daily admission volumes in a pediatric emergency department using the ARIMA method. \cite{afilal2016edflow} Developed a time series model to predict short-term and long-term patient flow, using a new patient classification (PE) that combines the CCMU (medical severity) and GEMSA (admission mode) classifications. \cite{chennupati2025aira} Used Generative Artificial Intelligence (GenAI) and Robotic Process Automation (RPA) to dynamically manage hospital operations through real-time predictive planning. \cite{bartek2019ml} Developed Linear Regression models and Machine Learning models (XGBoost) to improve the estimation of procedure duration. \cite{martinez2021surgical} Developed a solution based on four machine learning algorithms (Linear Regression, Support Vector Machines, Regression Trees, and Bagged Trees) for predicting the duration of surgical procedures. \cite{jeffrey2023los} Developed machine learning models (XGBoost with SMOTE) to predict outpatient surgery patients at risk of prolonged length of stay. 
This review includes research that uses heuristic and metaheuristic methods to model and solve planning and scheduling problems. \cite{mezouari2023thesis} Proposed heuristic and metaheuristic methods (PRH, IDCH, ALNS) for the dynamic scheduling of tasks and reactive planning of care pathways. \cite{kroer2018or} utilized a heuristic method (Standard Relax-and-Fix Algorithm, Proposed 2-Step Relax-and-Fix Variant) to build robust operating room schedules that minimize overtime and release unused capacity. \cite{akbarzadeh2020or}  employed a two-phase heuristic for surgical case planning and scheduling. This approach incorporates nurse rescheduling decisions and the assignment of nurses to specific patients, with the dual objectives of maximizing operating room department efficiency and optimizing operating room profit. \cite{aringhieri2015metaheuristic} used a Two-Level Metaheuristic (AORPP) to create a solution for the joint operating room planning and scheduling problem.
Based on a different approach to dynamic planning and scheduling, this review explores the use of AI technologies to reduce waiting times, shorten length of stay, and achieve other operational objectives. \cite{fallahpour2024or} utilized the ε-constraint method to minimize both idle time and patient waiting times. \cite{bouchlaghem2024surgery} employed a Q-learning algorithm to minimize total patient hospitalization duration (makespan) and average patient waiting time. \cite{lee2020rl} employed a deep reinforcement learning (RL) approach based on Deep Q-Networks (DQN) to minimize weighted patient waiting time and penalties for emergent patients in emergency departments. \cite{harzi2017ed} developed a Mixed-Integer Linear Programming (MILP) model, solved using the commercial solver IBM ILOG CPLEX Optimization Studio, to minimize total patient waiting time in the emergency department. \cite{assad2019ed} developed a discrete-event simulation model to minimize total labor costs while satisfying patient queue and waiting time constraints. \cite{marchesi2020physician} employed a Two-Stage Stochastic Programming approach to solve the physician scheduling and assignment problem in emergency departments (EDs). \cite{mahmoudzadeh2020robust} proposed a robust optimisation (RO) approach to minimise waiting time for each priority level. \cite{munavalli2020mas} developed a real-time intelligent scheduler using a Multi-Agent System (MAS) model, incorporating three algorithms for patient/resource scheduling and dynamic resource replanning, which allocates patients and resources based on real-time department status. \cite{creps2017outpatient} generated a decision tree analysis to develop a dynamic appointment scheduling procedure for an outpatient clinic.\\
Some review articles were used to inspire the methodology and are studied in this review. \cite{knight2023review} presented a systematic meta-narrative review to analyze the application of AI/ML in medical appointment scheduling, exploring real-world implementations of Artificial Intelligence (AI) and Machine Learning (ML) to optimize clinical appointment planning.

\section{Conclusion}
This short review has synthesized and analyzed the burgeoning application of Artificial Intelligence in Emergency Department planning and scheduling. The findings demonstrate that AI techniques spanning machine learning for prediction, optimization algorithms for resource allocation, and simulation for workflow design hold significant potential to enhance operational efficiency, reduce waiting times, and improve patient flow management.
However, this review also highlights several critical limitations in the current body of research. A primary constraint is the lack of generalizability of many studies. Numerous models are developed and validated on limited, context-specific datasets, often from a single type of ED or healthcare setting. This restricts their applicability across diverse emergency departments with varying patient demographics, acuity mixes, and resource constraints. Furthermore, a substantial number of proposed models are tested primarily on synthetically generated data rather than real-world operational data, raising questions about their practical robustness, performance, and trustworthiness in dynamic clinical environments.
Finally, while predictive analytics are well explored, the transition from prediction to prescriptive, real-time decision-support remains underexplored. Future research must therefore pivot towards developing integrated, hybrid frameworks that seamlessly combine accurate forecasting with robust optimization and simulation. Emphasis should be placed on creating adaptable tools that can generalize across different ED contexts, utilizing real-time data feeds for dynamic rescheduling and resource reallocation. The ultimate goal is to move beyond theoretical models towards the implementation of practical, trustworthy AI systems that empower clinicians and managers to proactively reduce patient wait times, shorten lengths of stay, and optimize staff and resource calendars, thereby building more resilient and efficient emergency care systems for the future.

\section*{Disclosure of Interests}
\begin{normalsize}
The authors do not have any conflict of interest with other entities or researchers.
\end{normalsize}

\bibliographystyle{unsrt}
\bibliography{references}

\end{document}
